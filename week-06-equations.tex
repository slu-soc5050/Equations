\documentclass{tufte-handout}

\usepackage{xcolor}
\usepackage{graphicx}

% set hyperlink attributes
\hypersetup{colorlinks}

\usepackage{amsmath}

% set image attributes:
\usepackage{graphicx}
\graphicspath{ {images/} }

% create environment for bottom paragraph:
\newenvironment{bottompar}{\par\vspace*{\fill}}{\clearpage}

% ============================================================

% define the title
\title{SOC 4930/5050: Week 06 Equations Quick \\Reference}
\author{Christopher Prener, Ph.D.}
\date{October 2\textsuperscript{nd}, 2017}
% ============================================================
\begin{document}
% ============================================================
\maketitle % generates the title
% ============================================================

\vspace{5mm}
\section{Standard Error}
\begin{equation}
\scalebox{2} {$ { \sigma }_{ \bar{ X } } = \frac{ {\sigma}_{x} }{ \sqrt{ n } } $}
\end{equation}

\vspace{5mm}
\section{Z-Score for Sample Means}
\begin{equation}
\scalebox{2} {$ z = \frac{\bar{x}-\mu}{\frac{\sigma}{\sqrt{n}}} $}
\end{equation}

\vspace{5mm}
\section{Simple Power Analysis}\marginnote{Use any of the two-tailed critical values' z-scores depending on how wide you want your interval.}
\begin{equation}
\scalebox{2} {$ \left( \frac{1.96\sigma}{\Delta} \right)^{2} $}
\end{equation}

\vspace{5mm}
\section{Predictive Interval}\marginnote{Use any of the two-tailed critical values' z-scores depending on how wide you want your interval.}
\begin{equation}
\scalebox{2} {$ \left(\mu-1.96\sigma, \mu+1.96\sigma \right)  $}
\end{equation}

\vspace{5mm}
\section{Predictive Interval for Sample Mean}\marginnote{Use any of the two-tailed critical values' z-scores depending on how wide you want your interval.}
\begin{equation}
\scalebox{2} {$ \left(\mu-1.96\frac{\sigma}{\sqrt{n}}, \mu+1.96\frac{\sigma}{\sqrt{n}}  \right) $}
\end{equation}


\vspace{5mm}
\section{Confidence Interval for Sample Mean}\marginnote{Use any of the two-tailed critical values' z-scores depending on how wide you want your interval.}
\begin{equation}
\scalebox{2} {$ \left(\bar{x}-1.96\frac{\sigma}{\sqrt{n}}, \bar{x}+1.96\frac{\sigma}{\sqrt{n}}  \right) $}
\end{equation}


% ============================================================
\end{document}