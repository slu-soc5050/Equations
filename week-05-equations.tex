\documentclass{tufte-handout}

\usepackage{xcolor}
\usepackage{graphicx}

% set hyperlink attributes
\hypersetup{colorlinks}

\usepackage{amsmath}

% set image attributes:
\usepackage{graphicx}
\graphicspath{ {images/} }

% create environment for bottom paragraph:
\newenvironment{bottompar}{\par\vspace*{\fill}}{\clearpage}

% ============================================================

% define the title
\title{SOC 4930/5050: Week 05 Equations Quick \\Reference}
\author{Christopher Prener, Ph.D.}
\date{September 25\textsuperscript{th}, 2017}
% ============================================================
\begin{document}
% ============================================================
\maketitle % generates the title
% ============================================================

\vspace{5mm}
\section{Binomial Distribution}
For the binomial distribution, let: \\
\noindent $n =$ number of independent trials \\
\noindent $p =$ probability of success in each trial

\paragraph{Mean}\mbox{}\\
\begin{equation}
\scalebox{2} {$ \mu = np $}
\end{equation}

\paragraph{Standard Deviation}\mbox{}\\
\begin{equation}
\scalebox{2} {$ \sigma =\sqrt { np\left( 1-p \right)  } $}
\end{equation}

\vspace{5mm}
\section{Poisson Distribution}
For the Poisson distribution, let: \\
\noindent $n =$ count of independent events \\
\noindent $p =$ probability of success in each event \\
\begin{equation}
\scalebox{2} {$ \lambda = np $}
\end{equation}

\vspace{5mm}
\section{Standard Normal Distribution}
\paragraph{Standardized Scores: Population}\mbox{}\\
\begin{equation}
\scalebox{2} {$ z = \frac{x-\mu}{\sigma} $}
\end{equation}

\paragraph{Standardized Scores: Sample}\mbox{}\\
\begin{equation}
\scalebox{2} {$ z = \frac{x-\bar{x}}{s} $}
\end{equation}

\vspace{5mm}
\section{Skew}\marginnote{Note that $sk$ is an abbreviation that I use in my classes to differentiate skew from standard deviation. There is no single accepted abbreviation for skew. Similarly, there are a number of equations in use to calculate skew; this is one that I teach because it simplifies some of the required calculations.}
\begin{equation}
\scalebox{2} {$ sk = \sqrt{n}\frac { \sum _{ i=1 }^{ n }{ \left( {x}_{i}-\bar{x} \right)^3 }  }{ \left( \sum _{ i=1 }^{ n }{ \left( { x }_{ i }-\bar{x} \right)^2 }  \right)^\frac{3}{2}  } $}
\end{equation}

\vspace{5mm}
\section{Kurtosis}\marginnote{Note that there are a number of accepted abbreviations for kurtosis including $k$. There are also a number of equations in use to calculate kurtosis. As with skew, this is one that I teach because it simplifies some of the required calculations.}
\begin{equation}
\scalebox{2} {$ k = n\frac { \sum _{ i=1 }^{ n }{ \left( {x}_{i}-\bar{x} \right)^4 }  }{ \left( \sum _{ i=1 }^{ n }{ \left( { x }_{ i }-\bar{x} \right)^2 }  \right)^2  } $}
\end{equation}
% ============================================================
\end{document}